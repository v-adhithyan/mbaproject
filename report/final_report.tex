\documentclass[a4paper, 14pt]{article}
\usepackage{float}
\usepackage{geometry}
\usepackage{graphicx}
\usepackage[utf8]{inputenc}
\usepackage{setspace}
\usepackage{lmodern}
\usepackage[hidelinks]{hyperref}

\linespread{2}
\newcommand\tab[1][1cm]{\hspace*{#1}}
\geometry{left=20mm,right=20mm,top=20mm,bottom=20mm}
\begin{document}
{
\pagenumbering{roman}
\fontfamily{ptm}\selectfont
\begin{center}	
\textbf{\fontsize{18}{2} \selectfont POST MERGER ANALYSIS OF CUSTOMER SATISFACTION AND LOYALTY - A STUDY ON RECENT MERGER OF ASSOCIATE BANKS OF SBI WITH ITSELF}\\
\tab \\
\textbf{\fontsize{14}{2} \selectfont A PROJECT REPORT}\\
\tab \\
\textbf{\fontsize{14}{2} \selectfont \emph{Submitted by}}\\
\tab \\
\tab \\
{\fontsize{16}{2} \selectfont
\textbf{ADHITHYAN V}}\\
{\fontsize{16}{2} \selectfont \textbf{(2016201002)}}\\
\tab \\
\textbf{\emph{\fontsize{14}{2} \selectfont in partial fulfillment for the award of the degree\\ of}}\\
\tab \\
\textbf{\fontsize{16}{2} \selectfont MASTER OF BUSINESS ADMINISTRATION}\\
\begin{figure}[H]
\centering
\includegraphics[scale=0.5]{anna_univ_logo.jpg}
\end{figure}
\tab \\
\textbf{\fontsize{14}{2} \selectfont COLLEGE OF ENGINEERING, GUINDY}\\
\tab \\
\textbf{\fontsize{16}{2} \selectfont ANNA UNIVERSITY : CHENNAI 600 025}\\
\tab \\
{\fontsize{14}{2} \selectfont MAY 2018}\\
\end{center}
	
% bonafide certificate
\newpage
\begin{center}
\textbf{\fontsize{18}{2} \selectfont ANNA UNIVERSITY : CHENNAI 600 025}\\
\tab \\
\textbf{\fontsize{16}{2} \selectfont BONAFIDE CERTIFICATE}\\
\end{center}
Certified that this project report \textbf{"POST MERGER ANALYSIS OF CUSTOMER SATISFACTION AND LOYALTY - A STUDY ON RECENT MERGER OF ASSOCIATE BANKS OF SBI WITH ITSELF"} is the bonafide work of \textbf{"ADHITHYAN V (2016201002)"} who carried out the project work under my supervision. Certified further, that to the best of my knowledge the work reported herein does not form part of any other project work or dissertation on the basis of which a degree or award was conferred on an earlier occasion on this or any other candidate.
\tab \\
\tab \\
\tab \\

\noindent \textbf{SIGNATURE OF HEAD OF THE DEPARTMENT} \hfill \hfill \textbf{SIGNATURE OF GUIDE}\\
\noindent Dr. L. Suganthi B.Tech., MBA., Ph.D., \hfill \hfill Dr.A.Thiruchelvi,\\
\noindent Professor \& Head, \hfill \hfill Asst.Professor,\\
\noindent Department of Management Studies, \hfill \hfill Department of Management Studies,\\
\noindent Anna University, \hfill \hfill Anna University,\\
\noindent Chennnai - 25. \hfill \hfill Chennai - 25.

%acknowledgement
\newpage
\begin{center}
\textbf{\fontsize{16}{2} \selectfont ACKNOWLEDGEMENT}\\
\end{center}
\par I would like to express my sincere gratitude to \textbf{Dr. L. Suganthi}, Head of the Department, and all beloved staffs of Department of Management Studies for instilling confidence in me to carry out this study and extending valuable guidance and encouragement from time to time, without which it would not have been possible to undertake and complete this project.  

\par I wish to thank my project guide \textbf{Dr. A. Thiruchelvi}, Assistant Professor, who has given her support, advice and guidance in completing the project successfully.

%abstract
\newpage
\begin{center}
\textbf{\fontsize{16}{2} \selectfont ABSTRACT}\\
\end{center}
\par Mergers typically involve two relatively equal companies making the mutually beneficial decision to become a single legal entity. They are different from acquisitions, which usually involve a larger company absorbing a smaller company, sometimes against the will of the smaller company?s management. Mergers are undertaken to improve long-term shareholder value and overall company performance. They are often done to reduce operating costs, improve market pentration and diversification.
\par A satisfied customer remains loyal and spreads positive word of mouth. Mergers and acquisitions often focus on financial aspects but rarely consider the customer facet of mergers. Studies show that 2/3 rd of mergers fail due to dissatisfied customers.A dissatisfied customer also switches the brand.Merger process are often done without considering the customers. Studies have found that more than half of all mergers fail to deliver the intended improvement in stakeholder value and that customer defections contribute to that high failure rate. Thus, a harmonious integration of the beliefs and values of a merging firm and the ability to integrate organisational cultures is more important to success than the financial or strategic factors. Customer switching also increases post merger. This study aims to find out loyalty and customer satisfaction post merger considering various factors such as demographics, brand image, psychological breach of contract etc taking the recent merger of associate banks of SBI with itself as a case study.

\newpage
\tableofcontents
\listoffigures
\listoftables

\newpage
\pagenumbering{arabic}
%introduction
\section{INTRODUCTION}
\par Public Sector Banks (PSBs) are banks where a majority stake (i.e. more than 50 \%) is held by a government. The shares of these banks are listed on stock exchanges. There are a total of 21 PSBs in India.

\subsection{LIST OF PUBLIC SECTOR BANKS IN INDIA}
\begin{enumerate}
\item Allahabad Bank
\item Andhra Bank
\item Bank of India
\item Bank of Baroda
\item Bank of Maharastra
\item Canara Bank
\item Central Bank of India
\item Corporation Bank
\item Dena Bank
\item Indian Bank
\item Indian Overseas Bank
\item Oriental Bank of Commerce
\item Punjab and Sindh Bank
\item Punjab National Bank
\item Syndicate Bank
\item UCO Bank
\item Union Bank of India
\item United Bank of India
\item Vijaya Bank
\end{enumerate}
\par SBI and IDBI often mentioned in this list are regarded as PSUs and not as nationalised banks themselves.

\subsection{FORMATION OF SBI AND ASSOCIATE BANKS}
\par The Central Government entered the banking business with the nationalization of the Imperial Bank of India in 1955. A 60\% stake was taken by the Reserve Bank of India and the new bank was named as the State Bank of India. The seven other state banks became the subsidiaries of the new bank in 1959 when the State Bank of India (Subsidiary Banks) Act, 1959 was passed under the Nehru government.

The seven associate banks of SBI are
\begin{enumerate}
\item State Bank of Bikaner and Jaipur (SBBJ)
\item State Bank of Patiala (SBBJ)
\item State Bank of Mysore (SBM
\item State Bank of Travancore (SBT)
\item State Bank of Hyderabad (SBH)
\item State Bank of Indore (SBI)
\item State Bank of Saurashtra (SBS)
\end{enumerate}

\subsection {SBI AND IT'S ASSOCIATES MERGER}
\par According to a gazette notification dated 22 February 2017 , the government said that all shares of these associate banks would cease to exist as individual entities and would merger with SBI. On April 1, 2017 SBT, SBM, SBH, SBBJ and SBP was merged with SBI. This merger has catapulted SBI to top 50 banks in world. It will join the league of top 50 banks in the workd in terms of assets. The total customer base of the bank will reach 37 crores with a branch network of around 24,000 and nearly 59,000 ATMs across the country. The merged entity will have a deposit base of more than Rs 26 lakh crore and advances level of INR 18.50 lakh crore.
\par Post merger, the bank will rationalise its branch network by relocating some of the branches to maximise reach. This will help the bank optimise its operations and improve profitability. As per the bank the merger will enhance productivity, mitigate geographical risks, increase operational efficiency and driver synergies across multiple dimensions while ensuring increased levels of customer delight.

\subsection {REASON FOR SBI AND IT'S ASSOCIATES MERGER}
\par \textbf{Govt. Aid to 1 Merged SBI Group :} Firstly the SBI and associates are one of the largest Govt. undertaking of the Central Govt. whom annual allocation of subsidy and contribution towards Bad Debt Recovery and Share Capital has to be made by the Indian Govt. There is practically no sense of giving aid to so many banks separately when it can be given to a single entity. Govt. Aid is for sure to be given to these banks and not just SBI and group but all the banks. So Govt. Aid to a single SBI merged bank will be much easier in terms of accountability.
\par \textbf{Bad loans and inability to recover :} SBI and group is the one of the largest banking sector entities who have crores and crores of Bad Loans which are not recoverable. Some entities Gross NPA has reached up to 20\%. Due to huge bad loans an internal corporate restructuring is required for all the associate group entities, otherwise in upcoming few years, few of them may even not survive in the market.
\par \textbf{Corporate restructuring :} Merger of the group entities of SBI is a way to restructure the Balance Sheet of the entities. Restructuring is required when the entities are facing financial crises or there is a possibility of the entity to not be able to meet out its existing liabilities. In corporate restructuring some liabilities are set off with realisation of assets. In this case, some entities liabilities will be sett off against the higher revalued assets of the other entities in order to make a good and attractive Balance Sheet Size of the merged entity.
\par \textbf{Bigger Bank :}By merging all the associate entities, SBI will become a much bigger and better bank as it will be catering to al large segment of customers as from its current position. It will be able to make many services convenient to the customers through a single bank rather than approaching other associated banks. It will have larger customer base, hence chances of earning good profitability over its deposits. It will have the advantage of Synergy with the associated banks. No high integration cost will be paid since the set-up is almost similar. It will have good asset portfolio. Allover, good report will be created amongst its customers.
\par \textbf{Better management :} Since it will become one big merged Bank, it will have only a management system rather than having different management set-up over the associate banks. Because of single management, efficiency and effectiveness of the business processes will be increased. Single circular will be issued for all the merged Banks for operational and management supervision. Better internal control and system processes will be carries on with all the merged banks
\par \textbf{Better increased recognition :} Those areas where SBi is not having branches but its associate banks are having, upon the merger being effected, the customer confidence and good report will be created because SBI is having a good report for all its customers but the other associate banks are not that good as the SBI. Also, they do not enjoy all those benefits as the SBI. Some sort of change in name from SBI associates to SBI will have a good market impression and will generate goodwill.

\subsection{ADVANTAGES OF MERGER}
\begin{itemize}
\item The most common reason for firms to enter into merger and acquisition is to merge their power and control over the markets.
\item Another advantage is Synergy that is the magic power that allow for increased value efficiencies of the new entity and it takes the shape of returns enrichment and cost savings.
\item Economies of scale is formed by sharing the resources and services. Union of 2 firm's leads in overall cost reduction giving a competitive advantage, that is feasible as a result of raised buying power and longer production runs.
\item Decrease of risk using innovative techniques of managing financial risk.
\item To become competitive, firms have to be compelled to be peak of technological developments and their dealing applications. By M\&A of a small business with unique technologies, a large company will retain or grow a competitive edge.
\item The biggest advantage is tax benefits. Financial advantages might instigate mergers and corporations will fully build use of tax- shields, increase monetary leverage and utilize alternative tax benefits 
\end{itemize}

\subsection{DISADVANTAGES OF MERGER}
\begin{itemize}
\item Loss of experienced workers aside from workers in leadership positions. This kind of loss inevitably involves loss of business understand and on the other hand that will be worrying to exchange or will exclusively get replaced at nice value.
\item As a result of M\&A, employees of the small merging firm may require exhaustive re-skilling.
\item Company will face major difficulties thanks to frictions and internal competition that may occur among the staff of the united companies. There is conjointly risk of getting surplus employees in some departments.
\item Merging two firms that are doing similar activities may mean duplication and over capability within the company that may need retrenchments.
\item Increase in costs might result if the right management of modification and also the implementation of the merger and acquisition dealing are delayed.
\item The uncertainty with respect to the approval of the merger by proper assurances.
\item In many events, the return of the share of the company that caused buyouts of other company was less than the return of the sector as a whole.
\end{itemize}

\subsection{PROBLEM STATEMENT}
\par A satisfied customer remains loyal and spreads positive word of mouth. Mergers and acquisitions often focus on financial aspects but rarely consider the customer facet of mergers. Studies show that 2/3 rd of mergers fail due to dissatisfied customers.A dissatisfied customer also switches the brand.Merger process are often done without considering the customers. Studies have found that more than half of all mergers fail to deliver the intended improvement in stakeholder value and that customer defections contribute to that high failure rate. Thus, a harmonious integration of the beliefs and values of a merging firm and the ability to integrate organisational cultures is more important to success than the financial or strategic factors. Customer switching also increases post merger. This study aims to find out loyalty and customer satisfaction post merger considering various factors such as demographics, brand image, psychological breach of contract etc taking the recent merger of associate banks of SBI with itself as a case study.

\subsection{NEED FOR THE STUDY}
\par Most of the merger effectiveness are assessed based on financial performance, synergy. Rarely the studies assess the effectiveness of merger from customer perspective.The success of merger depends not only on financial aspects but also creating value for its stakeholders. There has been lot of failed mergers because most of the mergers didn?t consider human aspects of merger. This study is important to find out level of service changes in banks post merger and their impact on customer satisfaction and loyalty due to service changes and psychological breach of contract. A psychological contract breach refers to subjective perception that other party has failed to adequately fulfill promised obligations. This study will also be useful to assess the effectiveness of merger from customer point of view.

\subsection{OBJECTIVES}
\subsubsection{PRIMARY OBJECTIVE}
\begin{itemize}
\item To find out customer satisfaction and loyalty post merger.
\end{itemize}
\subsubsection{SECONDARY OBJECTIVE}
\begin{itemize}
\item To find out benefits of merger from customer perspective.
\item To find out various issues faced by customer post merger.
\end{itemize}
\newpage
\section{CONCLUSION}

%references
\newpage
\section{REFERENCES}
\begin{enumerate}
\item \href{http://www.thehindu.com/business/Industry/sbi-five-associate-banks-bmb-merge-with-sbi/article17757316.ece}{http://www.thehindu.com/business/Industry/sbi-five-associate-banks-bmb-merge-with-sbi/article17757316.ece}
\item \href{https://en.wikipedia.org/wiki/Public_sector_banks_in_India}{https://en.wikipedia.org/wiki/Public-sector-banks-in-India}
\item \href{https://www.ukessays.com/essays/economics/advantages-and-disadvantages-of-mergers-and-acquisition-economics-essay.php}{https://www.ukessays.com/essays/economics/advantages-and-disadvantages-of-mergers-and-acquisition-economics-essay.php}
\item \href{https://www.quora.com/in/What-is-the-reason-behind-SBI-associate-banks-merger-What-could-be-its-pros-and-cons}{https://www.quora.com/in/What-is-the-reason-behind-SBI-associate-banks-merger-What-could-be-its-pros-and-cons}
\end{enumerate}
}
\end{document}